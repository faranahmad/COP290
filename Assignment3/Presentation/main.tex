% Copyright 2004 by Till Tantau <tantau@users.sourceforge.net>.
%
% In principle, this file can be redistributed and/or modified under
% the terms of the GNU Public License, version 2.
%
% However, this file is supposed to be a template to be modified
% for your own needs. For this reason, if you use this file as a
% template and not specifically distribute it as part of a another
% package/program, I grant the extra permission to freely copy and
% modify this file as you see fit and even to delete this copyright
% notice. 

\documentclass{beamer}
% \usepackage{subcaption}
% There are many different themes available for Beamer. A comprehensive
% list with examples is given here:
% http://deic.uab.es/~iblanes/beamer_gallery/index_by_theme.html
% You can uncomment the themes below if you would like to use a different
% one:
% \usetheme{AnnArbor}
% \usetheme{Antibes}
% \usetheme{Bergen}
% \usetheme{Berkeley}
% \usetheme{Berlin}
% \usetheme{Boadilla}
% \usetheme{boxes}
% \usetheme{CambridgeUS}
% \usetheme{Copenhagen}
% \usetheme{Darmstadt}
% \usetheme{default}
% \usetheme{Frankfurt}
% \usetheme{Goettingen}
% \usetheme{Hannover}
% \usetheme{Ilmenau}
% \usetheme{JuanLesPins}
% \usetheme{Luebeck}
\usetheme{Madrid}
% \usetheme{Malmoe}
% \usetheme{Marburg}
% \usetheme{Montpellier}
% \usetheme{PaloAlto}
% \usetheme{Pittsburgh}
% \usetheme{Rochester}
% \usetheme{Singapore}
% \usetheme{Szeged}
% \usetheme{Warsaw}

\title[COP 290]{Space Invaders}

\subtitle{COP290: Assignment 3}

\author[Faran \and Kabir \and Kartikeya \and Prateek]{Faran Ahmad \and Kabir Chhabra \and Kartikeya Gupta \and Prateek Verma \\
  2013CS10220 \and 2013CS50287 \and 2013CS10231 \and 2013CS10246}

\institute[IITD] % (optional, but mostly needed)
{
  Department of Computer Science and Engineering\\
  IIT Delhi
}

\date{March 16, 2015}
\subject{Design Practices}

\AtBeginSubsection[]
{
  % \begin{frame}<beamer>{Outline}
  %   \tableofcontents[currentsection,currentsubsection]
  % \end{frame}
}

\begin{document}

\begin{frame}
  \titlepage
\end{frame}

\begin{frame}{Objectives}{}
	Problem statement in brief
\end{frame}

\begin{frame}{Our choice}{Space Invaders}
    \begin{figure}[ht!]
      \centering
          \includegraphics[width=1.0\linewidth]{gameplay.png}
    \end{figure}
\end{frame}


\begin{frame}{Space Invaders}{Basic Game-play}
  \begin{itemize}
  	\item The player will control a space ship and shoot down aliens.
  	\item The aliens will shoot bullets at the players ship.
  	\item On getting hit by bullets the player will lose 1 life.
  	\item On destroying a large number of aliens, the player will get bonus lives.
  \end{itemize}
\end{frame}

\begin{frame}{Space Invaders}{Multi-player}
	\begin{block}{Co-op Mode}
	\begin{itemize}
		\item In co-op mode, the different players will team up to fight the aliens.
		\item The points scored by each will be combined together.
	\end{itemize}
	\end{block}
	\begin{block}{Competitive mode}
	\begin{itemize}
		\item Players will be put up against the same aliens but their scores will be separate. 
		\item The one who kills more aliens and / or survives the longest will get a higher score.
	\end{itemize}  
	\end{block}
\end{frame}

\begin{frame}{Space Invaders}{Scoring Scheme}
	\begin{block}{Lives}
		\begin{itemize}
			\item Each player will be given 3 lives.
			\item On getting hit by an alien bullet or colliding with an alien, a life will be lost.
			\item After killing 10 aliens in a row without any waste shot, a life will be awarded.
		\end{itemize}
	\end{block}
	
	\begin{block}{Scoring}
		\begin{itemize}
			\item On killing an alien a point would be avoided.
			\item On killing more and more aliens in a row, a multiplying factor associated with points would increase.
		\end{itemize}
	\end{block}
\end{frame}


\begin{frame}{Network Design}{}
	TODO: SOCCER
\end{frame}

\begin{frame}{Network Design}{Some more}
	TODO: SOCCER
\end{frame}

\begin{frame}{Network Design}{Network Outages}
	TODO: SOCCER \\
	Something about replacing player with AI player of same level till network is back. \\
	Also something on if the AI server goes down then a different user becomes the AI server.
\end{frame}

\begin{frame}{Artificial Intelligence}{Overview}
  The working of the enemy/opponent will be based on the concept of finite state machines where the enemy/ opponent will transition between particular states based on the situation. Different states define different modes of operation which include attacking, dodging or fleeing.
\end{frame}

\begin{frame}{Artificial Intelligence}{}
	\begin{block}{Enemy}
  			Difficulty Level : Three Difficulty levels: easy medium and hard. \\
			Enemy: Speed of enemy and frequency of bullets fired will be a function of difficulty.
 	\end{block}
	\begin{block}{Opponent} 
  			Accuracy of the opponent, frequency of bullets fired, and dodging ability of the opponent will be a function of difficulty.
	\end{block}
	\begin{block}{Incorporation}
		For games with simple entities, Entity pull systems work best where entities call on the AI system when they update themselves.
	\end{block}
\end{frame}

\begin{frame}{Time Line}{}
	  
\end{frame}

\begin{frame}
	\vfill
	\begin{center}
		\huge{Thank You}
	\end{center}
	\vfill
\end{frame}

\end{document}