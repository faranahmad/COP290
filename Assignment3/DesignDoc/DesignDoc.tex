\documentclass{article}
\usepackage[utf8]{inputenc}
\usepackage{listings}
\usepackage{color} 
\usepackage{titling}
\usepackage{graphicx}
\usepackage{titlepic}

\lstset{
	frame=tb, % draw a frame at the top and bottom of the code block
   	tabsize=4, % tab space width
   	showstringspaces=false, % don't mark spaces in strings
    numbers=left, % display line numbers on the left
    commentstyle=\color{red}, % comment color
    keywordstyle=\color{blue}, % keyword color
    stringstyle=\color{green} % string color
}


\title{\vspace*{\fill} \textbf{COP 290 Assignment 3}
	  \\ {\Large \textbf{Space Invaders}}
	  % \\  \vspace{3mm} \includegraphics{ddlogo.png}}
}
\author{
	\vspace{5mm} \includegraphics[width=5cm]{logo.png} \\
	 \textbf{Faran Ahmad}\\
	2013CS10220 \vspace{2mm} \\
	\textbf{Kabir Chhabra}\\ 
	2013CS50287 \vspace{2mm} \\
	\textbf{Kartikeya Gupta}\\ 
	2013CS10231 \vspace{2mm} \\
	\textbf{Prateek Kumar Verma}\\ 
	2013CS10246
}
\date{\vspace{3mm} \textbf{March 2015} \vspace*{\fill}}

\begin{document}
	\maketitle

	\newpage

	\tableofcontents

	\newpage

	\section{Objectives}
		About the assignment
	\section{Overall Design}
		\begin{enumerate}
			\item About the components and layers
		\end{enumerate}

	\section{Sub Components}
		% \begin{enumerate}

			\subsection{Back End}
				\subsubsection{Alien}
					\begin{lstlisting}[language=C++, caption={Class Parameters for Alien}]
class Alien
{
private:
	float XPos;
	float YPos;
	float Angle;
	Color ColorOfAlien;
	int Level;
	int PresentLives;
	int NumberBullets;
	int NumberMissiles;
	int AlienType;
};
					\end{lstlisting}
				\subsubsection{Ship}
					\begin{lstlisting}[language=C++, caption={Class Parameters for Ship}]
class Ship
{
private:
	float XPos;
	float YPos;
	float Angle;
	std::string Name;
	Color ColorOfShip;
	int Lives;
	int Score;
	int Multiplier;
	int Kills;
	int Id;
	int NumberBullets;
	int NumberMissiles;
	int AILevel;
};
					\end{lstlisting}
				\subsubsection{Color}
					\begin{lstlisting}[language=C++, caption={Class Parameters for Color}]
class Color
{
private:
	float R;
	float G;
	float B;
};
					\end{lstlisting}
				\subsubsection{Bullet}
					\begin{lstlisting}[language=C++, caption={Class Parameters for Bullet}]
class Bullet
{
private:
	float XPos;
	float YPos;
	float VelX;
	float VelY;
	Color ColorOfBullet;
	int ShipId;	
	bool TypeAI;
	bool TypePlayer;
};
					\end{lstlisting}
				\subsubsection{Board}
					\begin{lstlisting}[language=C++, caption={Class Parameters for Board}]
class Board
{
private:
	std::vector<Ship> VectorShips;
	std::vector<Bullet> VectorBullets;		
	std::vector<Alien> VectorAliens;
	double DimensionPosX;
	double DimensionPosY;
	double DimensionNegX;
	double DimensionNegY;	
};
					\end{lstlisting}
			\subsection{Artificial Intelligence}
			\subsection{Graphics}
			% \newline
			\subsection{Network Part}
	% \newline
	\section{Interaction amongst Sub Components}
		% \subsection{enumerate}
			\subsection{Back-end and UI}
			\subsection{Back-end and Network}
	\section{Testing Of Components}
		% \begin{enumerate}
			\subsection{General Unit Tests}
				% \newline
				\begin{lstlisting}[language=C++, caption={Class Parameters for Test}]
class Test
{
private:
	bool verbose;               //If test is to be conducted
	std::string description;    //String description of the test
	bool isPass;                //Boolean if the test has passed 
	void PrintPassFail(bool);   //Prints the status of the test
};
				\end{lstlisting}

				We will use the aforementioned class ``Test'' to perform unit tests on the different files created. This will ensure that all the functions work correctly against some tests.

			\subsection{Graphics}
			\subsection{Artificial Intelligence}
			\subsection{Network Component}
			\subsection{Overall Testing}

	\section{Extra Features}
		\subsection{Competitive Multi-player Mode}
		\subsection{3D Game-play}

\end{document}
