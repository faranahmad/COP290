\documentclass{article}
\usepackage[utf8]{inputenc}
\usepackage{listings}
\usepackage{color} 
\usepackage{titling}
\usepackage{graphicx}

\lstset{
	frame=tb, % draw a frame at the top and bottom of the code block
   	tabsize=4, % tab space width
   	showstringspaces=false, % don't mark spaces in strings
    numbers=left, % display line numbers on the left
    commentstyle=\color{red}, % comment color
    keywordstyle=\color{blue}, % keyword color
    stringstyle=\color{green} % string color
}

\title{\vspace*{\fill} \textbf{DeadDrop} \\
	\vspace{5mm} \includegraphics[width=7cm]{logo.png}}

\author{
	\textbf{Faran Ahmad}\\
	2013CS10220 \vspace{2mm} \\
	\textbf{Kartikeya Gupta}\\ 
	2013CS10231 \vspace{2mm} \\
	\textbf{Prateek Kumar Verma}\\ 
	2013CS10246
}
\date{\textbf{COP290: Design Practices} \vspace*{\fill}}

\begin{document}
	\maketitle

	\newpage

	\tableofcontents

	\newpage

	\section{Objectives}
	We have to build an on-line file management system ``Dead Drop'' . A server machine maintains the files of multiple users. The user should use a simple desktop application to login into the system. The content of user's account should remain synced with the server. 

	\section{Overall Design}
		\begin{enumerate}
			\item We will begin with creating different sub components like a File Transferring System, Credential Verifier, GUI part.
			\item Once the components are ready, we will Link this to the network and get basic functionality working on the local-host.
			\item Once the local interface is ready, we will take this to the web portal. We will use a server to store data and users will have to send queries to it
			\item Once the backend and front end is complete, we will link the two together.
		\end{enumerate}

	\section{Sub Components}
		% \begin{enumerate}

			\subsection{User Verification}
			\begin{lstlisting}[language=C++, caption={Class Parameters for User}]
class User
{
	private:
		std::string UserName;
		std::string PassWord;	
};
			\end{lstlisting}
			\begin{lstlisting}[language=C++, caption={Class Parameters for UserBase}]
class UserBase
{
private:
	std::unordered_map<std::string, std::string> UsersList;
};
			\end{lstlisting}fafaf
			The User Base is a hash table in which the keys are user-names and the stored values are passwords. When the credentials of the user are to be verified, the key is looked up in the table. Inserting users is also achieved easily using this model. The features which we will be provided to the user will be to verify credentials, add new users and change password. On the server, the credentials will be stored in an encrypted file which will be decrypted by the server program.

			\subsection{Files of User}

			We will use boost library to detect changes in files. For each file, the path of the file and last modified time of file is stored in a database.

			\begin{lstlisting}[language=C++, caption={Class Parameters for File History}]
class FileHistory
{
	private:
		std::string FolderLocation;
		int TimeOfData;
		std::vector< std::pair<std::string, int> > FileTimeBase;
};
			\end{lstlisting}

			Folder Location is the path of the synced folder. The parameter ``TimeOfData'' contains the system time at which the data detection was done. This will be used to determine if the server or client side file is newer and then do changes accordingly. ``FileTimeBase'' is a vector of a string and an integer. The string is the path of the file and the time is the time at which the file was last modified. In a nutshell, a socket represents a single connection between exactly two pieces of software. Sockets are bidirectional, meaning that either side of the connection is capable of both sending and receiving data. 
			
			% \newline
			\subsection{Network Managing Part}
				We will use sockets and socket programming to design and implement the network aspect of the assigment.Sockets allow applications to communicate using standard mechanisms built into network hardware and operating systems.
			\subsection{GUI interface} 
				The interface of the application would be designed using QtCreator. It would consists of the following mainwindows.
				\begin{itemize}
					\item \textbf{User Login}
						\newline
						Running the application would display a user login window. The users can access into their accounts by entering their user name and password in the respective fields. It also provides options for new user to signup for a new account in dead drop. In case, the user forgets his password, he can reset it using `forgot password' option. If the user name or password entered is wrong, a message box showing this message is displayed and the user can again enter the required informations to login to his account. 
					 
					\item \textbf{User Files}
						\newline
						Once the login procedure is complete, the user is directed to a new window. It contains the list of files of the user, both on the client and server side. The following buttons would be provided on the window for various functions
					\begin{itemize}
						\item \textit{sync :-} Clicking this button would sync the files on the user and the client side.
						\item \textit{Share :-} This button allows the user to share files with other users. Once the user has selected the files, clicking on this button would open a new dialog box, where the user can enter the name of the ones, with whom he wants to share the selected files. It also provide users with the options to set permissions for the shared files and folders.
						\item \textit{Delete from local :-} This button can be used to delete the selected files on the client side.
						\item \textit{Move To Drive :-} It can be used to move the seleted files on the client side to the server.
						\item \textit{Refresh :-} If there is some transfer of files between local data base on the client side, refresh button can be used to display the present ...
						\item \textit{Get :-} This button can be used to transfer the selected files on the server to the client side.
						\item \textit{Delete from Server :-} This button can be used to delete the selected files on the server.
						\item \textit{Exit :-} This button can be used to exit the application.
					\end{itemize} 
						
					\item \textbf{Server Side User Interface}
					\newline
					The window on the server side displays the list of all online users. It also provides the following buttons for different functions
					\begin{itemize}
						\item \textit{View All Users :-} It can be used to display the list of all the users.
						\item \textit{Remove A User :-} Clicking this button would open a new dialog box, displaying the list of the users. The users to be deleted can be selected from here for this operation.
						\item \textit{Files and Folders :-} It can be used to view all the files and folders stores on the server.
						\item \textit{Shut Down :-} This would shut down the server side window.
					\end{itemize}
				\end{itemize}

	% \newline
	\section{Interaction amongst Sub Components}
		% \subsection{enumerate}
			\subsection{User Authentication}
				% \newline a
				\begin{itemize}
					\item Client
						TODO soccer
					\item Server
						TODO soccer.
						\newline
						The network part mentioned above is linked with the user base file. The instruction to be performed is decoded to be a new user or credential verification. The data base of user names and passwords are accessed for this to take place and changes if needed are made accordingly to it.
				\end{itemize}
			\subsection{File Transfer}
				\begin{itemize}
					\item Files to Transfer
						TODO KG
					\item Transferring
						TODO soccer
				\end{itemize}	
			\subsection{File Sharing and Syncing}
				\begin{itemize}
					\item File changes
						TODO KG
					\item File Sharing and Permissions
						TODO KG
					\item File Syncing
						TODO KG
					\item File Syncing over network
						TODO KG
						TODO Soccer
				\end{itemize}
			\subsection{Front End and Back End}
				TODO KARTIKEYA 
		% \end{enumerate

	\section{Testing Of Components}
		% \begin{enumerate}
			\subsection{General Unit Tests}
				% \newline
				\begin{lstlisting}[language=C++, caption={Class Parameters for Test}]
class Test
{
	private:
		bool verbose;               //Variable if test is to be conducted
		std::string description;    //String description of the test
		bool isPass;                //Boolean if the test has passed 
		void PrintPassFail(bool);   //Prints the status of the test
};
				\end{lstlisting}

				We will use the aforementioned class ``Test'' to perform unit tests on the different files created. This will ensure that all the functions work correctly against some tests.

			\subsection{File Discovery}
				% \newline
				To test file discovery, a folder with different files will be used. The program will be run on this to obtain the list of files with their modified time and verified to check if it is in accordance with expectations. This will involve new files being created, files being modified and removed.
			\subsection{File Transferring}
				% \newline
				TODO SOCCER
			\subsection{UI Testing}
				% \newline
				%TODO FARAN write about individual components
				\begin{itemize}
					\item \textbf{User verification Testing}
						\newline
						To test user verification part, newusers were created in the backend. Then, their validity would be tested by typing their username and password in their respective fields to see if they are able to login to their account. The output of username and password entered were also displayed on the terminal to check our input manually. 
						TODO KG
					\item \textbf{New User Testing}
						\newline
						To test the newuser signup part, newusers would be added using signup button on the main window. The entered informations would then be checked in the back end by seeing the list of all the users and their informations. 
						TODO KG
					\item \textbf{File Managing Part}
						\newline
						The user files part would be checked manually by comparing that list of files of a user, displayed in his account is same as the list of files that can obtained by browsing into the user folder. Similarly, list of files of a user on the server side being displayed in his account should be same as the one obtained by browsing into the server.
						TODO KG			
				\end{itemize}
			\subsection{Overall Testing}
				% \newline
				TODO KG once above 2 are done
		% \end{enumerate}

	\section{Extra Features}
		% \begin{enumerate}
		\subsection{Keeping Files On-line only.}
				% \newline
				We will give the user an option to keep a file on the cloud only and not on his PC so that lesser space will be used on the users PC. The user can get these files or remove them from the cloud if he wishes.
		\subsection{Intelligent UI for server}
			% \newline
			TODO KG
			Write about showing transfer speed and stuff
		\subsection{Allowing Incomplete Downloads}
			Allowing the user to resume downloads of files from where he left off if the connection with the server breaks in the middle.
		\subsection{De-duplication}
		% \end{enumerate}

\end{document}
