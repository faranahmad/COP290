\documentstyle[11 pt]{article}
%Default margins are too wide all the way around.I reset them here 
\setlength{\topmargin}{-.5in}
\setlength{\textheight}{9in}
\setlength{\oddsidemargin}{.125in}
\setlength{\textwidth}{6.25in}
\begin{document}
\title
{COMMON DESIGN DOCUMENT}
\author
{Harsh Gupta \\ 2009CS10191 \ and  Amogh Suman \\ 2009CS10179}
\renewcommand{\today}{September 15, 2010}
\maketitle 
\begin{abstract}
This document contains information about all common aspects to the multiplayer
  ping pong game
\end{abstract}
\pagebreak 
\section{State Abstraction}
\subsection{What is the state that is communicated ?}
% %this answer is not at all clear
%   At any instant in the game the state that is communicated must contain  sufficient information about the gameplay so as to make it possible for each
%   receiver to display the present game state to the player and also make the  necessay changes in the data that it stores.  Each client involved vin the gameplay already has some information about  each of the variables that determine the game state at time t.  For the communication at $t +  \Delta t$ from computer 1 to computer 2 just the diff is sent. 
 If a server maintains state, it means that the server maintains
information about all of the current connections with its clients and the communications sent between them. Otherwise, the server application has no method to determine if the client
intends to communicate further, is done communicating, is waiting for a
response, or has experienced an error.
We can define our states broadly in the following categories:
\begin{enumerate}
 \item Initial State: When the server begins a session for the client,and assigns him a player id ,conveys basic information about the gameplay etc.
 \item Active State: When the client is playing the game on the server and so is ``active``.
\item Disconnected State: When the client was playing the game earlier but is disconnected from the network due to network issues.
\item Aborted State:When the client was playing the game and then he aborted his application on his terminal in using a SIGINT .
\end{enumerate}

  \subsection{What happens if a host goes down ?}
  Say host $ M $ goes down.  Now the server shall have no information as to the activities of the player  on $ M $.  However the gameplay for the remaining of the players shall not be disturbed  by this activity.  For the remainder of the players the gameplay shall continue with the player
  on machine $ M $ being replaced by a player controlled by the computer. This automated pseudo player shall have the same probability of hitting the  ball with the paddle as the human playing on computer $M$. 
  \footnote { For each human player we shall be computing the probability with which he/she successfully hits the ball and storing it as an attribute of the player.}
  \footnote{For an automated player the probability with which it hits the ball shall be   an attribute which the user can modify via appropriate menu}
  \subsection{What happens if a host is temporarily unavailable ?}
  Say host $N$ is temporarily available,  then we shall replace host $N$ by an automated player having the same hit  probability as the player who was playing on $N$.  His score shall be edited as per his shots and so the gameplay shall not be  affected at all for the remainder of the players.
  However as soon as the host shall be available,  the human on $N$ shall take up the position where the automated player left  him. 
  \pagebreak 
  \section{Communication over the network}
  \subsection{Protocol}
TCP\textbackslash IP protocol over IPv4 will be used for communication between multiple computers
%tcp uses a sliding window mechanism to send packages
%it is very much like sending the message via a stream
\subsection{Algorithm}
Let the total number of players be $N$
We shall be using multiple sockets for the server and a socket per player on every client computer.The server shall have a total  of $N+1$ sockets.
%Why this extra socket?
 % The connections will be put into a queue on a first-come, first-served basis. If we didn’t use an extra socket and were busy handling a request, then a second client would be unable to connect and would get a “connection refused”error message. By using two sockets, we can let our server handle its response while lining up other connections to be handled as soon as it’s done, without returning an error. So our server can handle more than one connection simultaneosly.
\footnote{With the extra socket, the server can handle a request from a client while still accepting more connections from other clients.The one extra socket shall be closed as soon as the initialisation process is complete}
Steps to be performed for the initialisation of communication across the network:
\begin{enumerate}
 \item On the server computer the application is started and it negotiates with the Operating system over the port to be used.
\item The application begins listening for incoming requests over the network( ie it uses a passive socket )
 \item Work with the sockaddr\_in structure:
\footnote{this contains both the IP adresses and protocol port number.}
\item setsocketopt will be used to set settings to keep socket alive for the entire game \footnote{linger is a parameter that determines whether a socket waits to see that all data
is read once the other end of the communication closes.
}We shall keep the socket open until the connection at the other end is closed by defining l\_onoff to be zero.
 \end{enumerate}
 \subsection{sending data}
Since TCP sends data as a stream we shall add markers to the beginning and end of each message to alienate it from the previous and next message.

%TCP DATAGRAM IS also called a segment. It is the thing that does all the work of advertising window size ,sending,receiving and authenticating over network
\pagebreak 
\section{Regression Testing}
\footnote{separation of the main game and preliminary input
The main game and the inputs that need to be taken from the user will be separate programs. The main game will basically take input via the command-line arguements (or 'argc').This will allow automation in the testing process .}

\subsection{Use of preprocessor macros :}
Preprocessor macros will be used to selectively compile code.
  We will have a macro NDEBUG which will be defined if the application is
  running in the non - debug mode.
  \subsection{Use of assert function :}
The assert function will be used to effectively trackdown bugs and also checkfor invalid conditions at several steps in the code. 
    \footnote{  Assert function is disabled if the macro NDEBUG is defined}
\begin{enumerate}
\item{ check if any of the thread id 's are NULL.}
\item{check if any coordinate that is being displayed is outside the game field.}
\end{enumerate}
\subsection{Automated tests:}
\footnote{The protocols used to transfer any files between connected machines will be ftp (port 21) and telnet (port 23)}
\begin{enumerate}
\item {Indefinitely carrying on game}
\begin{enumerate}
	\item {two player game}
	\item{ players can be on the same or different computer}
	\item {initial velocity of the ball in the x-direction is kept zero and y direction velocity is set arbitrarily.}
	\item{Expected Result: ball never goes out of the court if the initial y coordinate of the ball lies between the extreme y coordinates of the 
	      two paddles}
	\footnote{If the behaviour is expected to continue for an indefinite period of time,then we would check that the behaviour of the system is the same as expected for a fixed period of time}
	\end{enumerate}
\item{Ball never touches paddle}
	\begin{enumerate}
	\item {2 players}
	\item {players can be on the same or different computer.}
	\item {The ball is given a random velocity in the x direction .}
	\item {Expected Result:Ball never hits paddle.}
	\end{enumerate}
\item{Imperfect player vs perfect players:}
	\begin{enumerate}
	\item {2,3,4 player game}
	\item{number of balls can be 1,2,3}
	\item {Players can be on the same computer or different computers.Different combinations presesnt.}
	\item {Expected Result: The imperfect player loses}
	\end{enumerate}
\end{enumerate}
\end{document}
